\HeaderA{Manitoba.lakes}{The Nine Largest Lakes in Manitoba}{Manitoba.lakes}
\keyword{datasets}{Manitoba.lakes}
\begin{Description}\relax
The \code{Manitoba.lakes} data frame has 9 rows and 2 columns.
The areas and elevations of the nine largest lakes in
Manitoba, Canada.  The geography of Manitoba (a relatively
flat province) can be divided crudely into three main
areas: a very flat prairie in the south which is at a 
relatively high elevation, a middle region consisting
of mainly of forest and Precambrian rock, and a northern
region which drains more rapidly into Hudson
Bay.  All water in Manitoba, which does not evaporate, eventually drains 
into Hudson Bay.
\end{Description}
\begin{Usage}
\begin{verbatim}Manitoba.lakes\end{verbatim}
\end{Usage}
\begin{Format}\relax
This data frame contains the following columns:
\describe{
\item[elevation] a numeric vector consisting of the elevations
of the lakes (in meters)
\item[area] a numeric vector consisting of the areas of
the lakes (in square kilometers)
}
\end{Format}
\begin{Source}\relax
The CANSIM data base at Statistics Canada.
\end{Source}
\begin{Examples}
\begin{ExampleCode}
plot(Manitoba.lakes)
plot(Manitoba.lakes[-1,])
\end{ExampleCode}
\end{Examples}

