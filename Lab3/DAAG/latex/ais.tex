\HeaderA{ais}{Australian athletes data set}{ais}
\keyword{datasets}{ais}
\begin{Description}\relax
These data were collected in a study of how 
data on various characteristics of the bloood varied with sport
body size and sex of the athlete.
\end{Description}
\begin{Usage}
\begin{verbatim}data(ais)\end{verbatim}
\end{Usage}
\begin{Format}\relax
A data frame with 202 observations on the following 13 variables.
\describe{
\item[rcc] red blood cell count, in \eqn{10^{12} l^{-1}}{}
\item[wcc] while blood cell count, in \eqn{10^{12}}{} per liter
\item[hc] hematocrit, percent
\item[hg] hemaglobin concentration, in g per decaliter
\item[ferr] plasma ferritins, ng \eqn{dl^{-1}}{}
\item[bmi] Body mass index, kg \eqn{cm^{-2} 10^2}{}
\item[ssf] sum of skin folds
\item[pcBfat] percent Body fat
\item[lbm] lean body mass, kg
\item[ht] height, cm
\item[wt] weight, kg
\item[sex] a factor with levels \code{f} \code{m}
\item[sport] a factor with levels \code{B\_Ball} \code{Field} 
\code{Gym} \code{Netball} \code{Row} \code{Swim} \code{T\_400m} 
\code{T\_Sprnt} \code{Tennis} \code{W\_Polo}
}
\end{Format}
\begin{Details}\relax
Do blood hemoglobin concentrations of athletes in endurance-related
events differ from those in power-related events?
\end{Details}
\begin{Source}\relax
These data were the basis for the analyses that are reported in
Telford and Cunningham (1991).
\end{Source}
\begin{References}\relax
Telford, R.D. and Cunningham, R.B. 1991.  Sex, sport and
body-size dependency of hematology in highly trained athletes.
Medicine and Science in Sports and Exercise 23: 788-794.
\end{References}

