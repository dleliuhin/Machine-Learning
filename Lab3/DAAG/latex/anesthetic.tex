\HeaderA{anesthetic}{Anesthetic Effectiveness}{anesthetic}
\keyword{datasets}{anesthetic}
\begin{Description}\relax
Thirty patients were given an anesthetic agent maintained
at a predetermined level (conc) for 15 minutes before making
an incision.  It was then noted whether the patient
moved, i.e. jerked or twisted.
\end{Description}
\begin{Usage}
\begin{verbatim}anesthetic\end{verbatim}
\end{Usage}
\begin{Format}\relax
This data frame contains the following columns:
\describe{
\item[move] a binary numeric vector coded for 
patient movement (0 = no movement, 1 = movement)
\item[conc] anesthetic concentration
\item[logconc] logarithm of concentration
\item[nomove] the complement of move
}
\end{Format}
\begin{Details}\relax
The interest is in estimating
how the probability of jerking or twisting varies with
increasing concentration of the anesthetic agent.
\end{Details}
\begin{Source}\relax
unknown
\end{Source}
\begin{Examples}
\begin{ExampleCode}
print("Logistic Regression - Example 8.1.4")

z <- table(anesthetic$nomove, anesthetic$conc)
tot <- apply(z, 2, sum)         # totals at each concentration
prop <- z[2,  ]/(tot)           # proportions at each concentration
oprop <- sum(z[2,  ])/sum(tot)  # expected proportion moving if concentration had no effect
conc <- as.numeric(dimnames(z)[[2]])
plot(conc, prop, xlab = "Concentration", ylab = "Proportion", xlim = c(.5,2.5),
    ylim = c(0, 1), pch = 16)
chw <- par()$cxy[1]
text(conc - 0.75 * chw, prop, paste(tot), adj = 1)
abline(h = oprop, lty = 2)

pause()

anes.logit <- glm(nomove ~ conc, family = binomial(link = logit),
  data = anesthetic)
anova(anes.logit)
summary(anes.logit)

\end{ExampleCode}
\end{Examples}

