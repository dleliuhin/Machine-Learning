\HeaderA{antigua}{Averages by block of yields for the Antigua Corn data}{antigua}
\keyword{datasets}{antigua}
\begin{Description}\relax
These data frames have yield averages by blocks (parcels). The
\code{ant111b} data set is a subset of this.
\end{Description}
\begin{Usage}
\begin{verbatim}antigua\end{verbatim}
\end{Usage}
\begin{Format}\relax
A data frame with 324 observations on 7 variables.
\describe{
\item[id] a numeric vector
\item[site] a factor with 8 levels.
\item[block] a factor with levels \code{I} \code{II} \code{III} \code{IV}
\item[plot] a numeric vector
\item[trt] a factor consisting of 12 levels
\item[ears] a numeric vector; note that -9999 is used as a missing value code.
\item[harvwt] a numeric vector; the average yield
}
\end{Format}
\begin{Source}\relax
Andrews DF; Herzberg AM, 1985. Data. A Collection of Problems from
Many Fields for the Student and Research Worker. Springer-Verlag.
(pp. 339-353)
\end{Source}

