\HeaderA{austpop}{Population figures for Australian States and Territories}{austpop}
\keyword{datasets}{austpop}
\begin{Description}\relax
Population figures for Australian states and territories for 1917, 1927,
..., 1997.
\end{Description}
\begin{Usage}
\begin{verbatim}austpop\end{verbatim}
\end{Usage}
\begin{Format}\relax
This data frame contains the following columns:
\describe{
\item[year] a numeric vector
\item[NSW] New South Wales population counts 
\item[Vic] Victoria population counts
\item[Qld] Queensland population counts
\item[SA] South Australia population counts
\item[WA] Western Australia population counts
\item[Tas] Tasmania population counts
\item[NT] Northern Territory population 
counts
\item[ACT] Australian Capital Territory 
population counts
\item[Aust] Population counts for
the whole country
}
\end{Format}
\begin{Source}\relax
Australian Bureau of Statistics
\end{Source}
\begin{Examples}
\begin{ExampleCode}
print("Looping - Example 1.7")

growth.rates <- numeric(8)
for (j in seq(2,9)) {
    growth.rates[j-1] <- (austpop[9, j]-austpop[1, j])/austpop[1, j] }
growth.rates <- data.frame(growth.rates)
row.names(growth.rates) <- names(austpop[c(-1,-10)])
  # Note the use of row.names() to name the rows of the data frame
growth.rates

pause()
print("Avoiding Loops - Example 1.7b")

sapply(austpop[,-c(1,10)], function(x){(x[9]-x[1])/x[1]})

pause()
print("Plot - Example 1.8a")
attach(austpop)
plot(year, ACT, type="l") # Join the points ("l" = "line")
detach(austpop)

pause()
print("Exerice 1.12.9")
attach(austpop)
oldpar <- par(mfrow=c(2,4))  
for (i in 2:9){
plot(austpop[,1], log(austpop[, i]), xlab="Year",
    ylab=names(austpop)[i], pch=16, ylim=c(0,10))}
par(oldpar) 
detach(austpop)

\end{ExampleCode}
\end{Examples}

