\HeaderA{biomass}{Biomass Data}{biomass}
\keyword{datasets}{biomass}
\begin{Description}\relax
The \code{biomass} data frame has 135 rows and 8 columns.  The
\code{rainforest} data frame is a subset of this one.
\end{Description}
\begin{Usage}
\begin{verbatim}biomass\end{verbatim}
\end{Usage}
\begin{Format}\relax
This data frame contains the following columns:
\describe{
\item[dbh] a numeric vector
\item[wood] a numeric vector
\item[bark] a numeric vector
\item[fac26] a factor with 3 levels
\item[root] a numeric vector
\item[rootsk] a numeric vector
\item[branch] a numeric vector
\item[species] a factor with levels
\code{Acacia mabellae},
\code{C. fraseri},
\code{Acmena smithii},
\code{B. myrtifolia} 
}
\end{Format}
\begin{Source}\relax
J. Ash, Australian National University
\end{Source}
\begin{References}\relax
Ash, J. and Helman, C. (1990) Floristics and vegetation
biomass of a forest catchment, Kioloa, south coastal N.S.W.
Cunninghamia, 2: 167-182.
\end{References}

