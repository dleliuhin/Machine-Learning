\HeaderA{bomsoi}{Southern Oscillation Index Data}{bomsoi}
\keyword{datasets}{bomsoi}
\begin{Description}\relax
The Southern Oscillation Index (SOI) is the difference in barometric  
pressure at sea level between Tahiti and Darwin.  Annual SOI and
Australian rainfall data, for the years 1900-2001, are given.
Australia's annual mean rainfall is an area-weighted average of the total 
annual precipitation at approximately 370 rainfall stations 
around the country.
\end{Description}
\begin{Usage}
\begin{verbatim}bomsoi\end{verbatim}
\end{Usage}
\begin{Format}\relax
This data frame contains the following columns:
\describe{
\item[Year] a numeric vector
\item[Jan] average January SOI values for each year
\item[Feb] average February SOI values for each year
\item[Mar] average March SOI values for each year
\item[Apr] average April SOI values for each year
\item[May] average May SOI values for each year
\item[Jun] average June SOI values for each year
\item[Jul] average July SOI values for each year
\item[Aug] average August SOI values for each year
\item[Sep] average September SOI values for each year
\item[Oct] average October SOI values for each year
\item[Nov] average November SOI values for each year
\item[Dec] average December SOI values for each year
\item[SOI] a numeric vector consisting of average annual SOI
values
\item[avrain] a numeric vector consisting of a weighted average annual
rainfall at a large number of Australian sites
\item[NTrain] Northern Territory rain
\item[northRain] north rain
\item[seRain] southeast rain
\item[eastRain] east rain
\item[southRain] south rain
\item[swRain] southwest rain
}
\end{Format}
\begin{Source}\relax
Australian Bureau of Meteorology web pages:

http://www.bom.gov.au/climate/change/rain02.txt and
http://www.bom.gov.au/climate/current/soihtm1.shtml
\end{Source}
\begin{References}\relax
Nicholls, N., Lavery, B., Frederiksen, C.\ and Drosdowsky, W. 1996.
Recent apparent changes in relationships between the El Nino --
southern oscillation and Australian rainfall and temperature.
Geophysical Research Letters 23: 3357-3360.
\end{References}
\begin{Examples}
\begin{ExampleCode} 
plot(ts(bomsoi[, 15:14], start=1900),
     panel=function(y,...)panel.smooth(1900:2005, y,...))
pause()

# Check for skewness by comparing the normal probability plots for 
# different a, e.g.
par(mfrow = c(2,3))
for (a in c(50, 100, 150, 200, 250, 300))
qqnorm(log(bomsoi[, "avrain"] - a))
  # a = 250 leads to a nearly linear plot

pause()

par(mfrow = c(1,1))
plot(bomsoi$SOI, log(bomsoi$avrain - 250), xlab = "SOI",
     ylab = "log(avrain = 250)")
lines(lowess(bomsoi$SOI)$y, lowess(log(bomsoi$avrain - 250))$y, lwd=2)
  # NB: separate lowess fits against time
lines(lowess(bomsoi$SOI, log(bomsoi$avrain - 250)))
pause()

xbomsoi <-
  with(bomsoi, data.frame(SOI=SOI, cuberootRain=avrain^0.33))
xbomsoi$trendSOI <- lowess(xbomsoi$SOI)$y
xbomsoi$trendRain <- lowess(xbomsoi$cuberootRain)$y
rainpos <- pretty(bomsoi$avrain, 5)
with(xbomsoi,
     {plot(cuberootRain ~ SOI, xlab = "SOI",
           ylab = "Rainfall (cube root scale)", yaxt="n")
     axis(2, at = rainpos^0.33, labels=paste(rainpos))
## Relative changes in the two trend curves
     lines(lowess(cuberootRain ~ SOI))
     lines(lowess(trendRain ~ trendSOI), lwd=2)
  })
pause()

xbomsoi$detrendRain <-
  with(xbomsoi, cuberootRain - trendRain + mean(trendRain))
xbomsoi$detrendSOI <-
  with(xbomsoi, SOI - trendSOI + mean(trendSOI))
oldpar <- par(mfrow=c(1,2), pty="s")
plot(cuberootRain ~ SOI, data = xbomsoi,
     ylab = "Rainfall (cube root scale)", yaxt="n")
axis(2, at = rainpos^0.33, labels=paste(rainpos))
with(xbomsoi, lines(lowess(cuberootRain ~ SOI)))
plot(detrendRain ~ detrendSOI, data = xbomsoi,
  xlab="Detrended SOI", ylab = "Detrended rainfall", yaxt="n")
axis(2, at = rainpos^0.33, labels=paste(rainpos))
with(xbomsoi, lines(lowess(detrendRain ~ detrendSOI)))
pause()

par(oldpar)
attach(xbomsoi)
xbomsoi.ma0 <- arima(detrendRain, xreg=detrendSOI, order=c(0,0,0))
# ordinary regression model

xbomsoi.ma12 <- arima(detrendRain, xreg=detrendSOI,
                      order=c(0,0,12))
# regression with MA(12) errors -- all 12 MA parameters are estimated
xbomsoi.ma12
pause()

xbomsoi.ma12s <- arima(detrendRain, xreg=detrendSOI,
                      seasonal=list(order=c(0,0,1), period=12))
# regression with seasonal MA(1) (lag 12) errors -- only 1 MA parameter
# is estimated
xbomsoi.ma12s
pause()

xbomsoi.maSel <- arima(x = detrendRain, order = c(0, 0, 12),
                        xreg = detrendSOI, fixed = c(0, 0, 0,
                        NA, rep(0, 4), NA, 0, NA, NA, NA, NA),
                        transform.pars=FALSE)
# error term is MA(12) with fixed 0's at lags 1, 2, 3, 5, 6, 7, 8, 10
# NA's are used to designate coefficients that still need to be estimated
# transform.pars is set to FALSE, so that MA coefficients are not
# transformed (see help(arima))

detach(xbomsoi)
pause()

Box.test(resid(lm(detrendRain ~ detrendSOI, data = xbomsoi)),
          type="Ljung-Box", lag=20)

pause()

attach(xbomsoi)
 xbomsoi2.maSel <- arima(x = detrendRain, order = c(0, 0, 12),
                         xreg = poly(detrendSOI,2), fixed = c(0,
                         0, 0, NA, rep(0, 4), NA, 0, rep(NA,5)),
                         transform.pars=FALSE)
 xbomsoi2.maSel
qqnorm(resid(xbomsoi.maSel, type="normalized"))
detach(xbomsoi)

\end{ExampleCode}
\end{Examples}

