\HeaderA{bounce}{Separate plotting positions for labels, to avoid overlap}{bounce}
\keyword{utilities}{bounce}
\begin{Description}\relax
Return univariate plotting positions in which neighboring points are
separated, if and as necessary, so that they are the specified minimum
distance apart.
\end{Description}
\begin{Usage}
\begin{verbatim}
bounce(y, d, log = FALSE)
\end{verbatim}
\end{Usage}
\begin{Arguments}
\begin{ldescription}
\item[\code{y}] A numeric vector of plotting positions
\item[\code{d}] Minimum required distance between neighboring positions
\item[\code{log}] \code{TRUE} if values are will be plotted on a logarithmic scale.
\end{ldescription}
\end{Arguments}
\begin{Details}\relax
The centroid(s) of groups of points that are moved relative to each
other remain the same.
\end{Details}
\begin{Value}
A vector of values such that, when plotted along a line, neighboring
points are the required minimum distance apart.
\end{Value}
\begin{Note}\relax
If values are plotted on a logarithmic scale, \code{d} is the required
distance apart on that scale. If a base other than 10 is required, set
\code{log} equal to that base.  (Note that base 10 is the default for
\code{plot} with \code{log=TRUE}.)
\end{Note}
\begin{Author}\relax
John Maindonald
\end{Author}
\begin{SeeAlso}\relax
See also \code{\LinkA{onewayPlot}{onewayPlot}}
\end{SeeAlso}
\begin{Examples}
\begin{ExampleCode}
bounce(c(4, 1.8, 2, 6), d=.4)
bounce(c(4, 1.8, 2, 6), d=.1, log=TRUE)
\end{ExampleCode}
\end{Examples}

