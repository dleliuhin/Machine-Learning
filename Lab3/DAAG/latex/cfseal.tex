\HeaderA{cfseal}{Cape Fur Seal Data}{cfseal}
\keyword{datasets}{cfseal}
\begin{Description}\relax
The \code{cfseal} data frame has 30 rows and 11 columns consisting
of weight measurements for various organs taken from 30 Cape Fur
Seals that died as an unintended consequence of commercial fishing.
\end{Description}
\begin{Usage}
\begin{verbatim}cfseal\end{verbatim}
\end{Usage}
\begin{Format}\relax
This data frame contains the following columns:
\describe{
\item[age] a numeric vector
\item[weight] a numeric vector
\item[heart] a numeric vector
\item[lung] a numeric vector
\item[liver] a numeric vector
\item[spleen] a numeric vector
\item[stomach] a numeric vector
\item[leftkid] a numeric vector
\item[rightkid] a numeric vector
\item[kidney] a numeric vector
\item[intestines] a numeric vector
}
\end{Format}
\begin{Source}\relax
Stewardson, C.L., Hemsley, S., Meyer, M.A., Canfield,
P.J. and Maindonald, J.H. 1999.  Gross and microscopic visceral
anatomy of the male Cape fur seal, Arctocephalus pusillus pusillus
(Pinnepedia: Otariidae), with reference to organ size and growth.
Journal of Anatomy (Cambridge) 195: 235-255.  (WWF project ZA-348)
\end{Source}
\begin{Examples}
\begin{ExampleCode}
print("Allometric Growth - Example 5.7")

cfseal.lm <- lm(log(heart) ~ log(weight), data=cfseal); summary(cfseal.lm)
plot(log(heart) ~ log(weight), data = cfseal, pch=16, xlab = "Heart Weight (g, log scale)", 
ylab = "Body weight (kg, log scale)", axes=FALSE)
heartaxis <- 100*(2^seq(0,3))
bodyaxis <- c(20,40,60,100,180)
axis(1, at = log(bodyaxis), lab = bodyaxis)
axis(2, at = log(heartaxis), lab = heartaxis)
box()
abline(cfseal.lm)
\end{ExampleCode}
\end{Examples}

