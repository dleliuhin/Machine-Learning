\HeaderA{cottonworkers}{Occupation and wage profiles of British cotton workers}{cottonworkers}
\keyword{datasets}{cottonworkers}
\begin{Description}\relax
Numbers are given in different categories of worker, in each of
two investigations.  The first source of information is the
Board of Trade Census that was conducted on 1886.  The second
is a relatively informal survey conducted by US Bureau of
Labor representatives in 1889, for use in official reports.
\end{Description}
\begin{Usage}
\begin{verbatim}data(cottonworkers)\end{verbatim}
\end{Usage}
\begin{Format}\relax
A data frame with 14 observations on the following 3 variables.
\describe{
\item[census1886] Numbers of workers in each of 14 different
categories, according to the Board of Trade wage census that
was conducted in 1886
\item[survey1889] Numbers of workers in each of 14 different
categories, according to data collected in 1889 by the US Bureau
of Labor, for use in a report to the US Congress and House of
Representatives
\item[avwage] Average wage, in pence, as estimated in the US Bureau
of Labor survey
}
\end{Format}
\begin{Details}\relax
The data in \code{survey1889} were collected in a relatively informal
manner, by approaching individuals on the street.  Biases might
therefore be expected.
\end{Details}
\begin{Source}\relax
United States congress, House of Representatives, Sixth Annual Report
of the Commissioner of Labor, 1890, Part III, Cost of Living
(Washington D.C. 1891); idem.,  Seventh Annual Report
of the Commissioner of Labor, 1891, Part III, Cost of Living
(Washington D.C. 1892)

Return of wages in the principal textile trades of the United Kingdom,
with report therein. (P.P. 1889, LXX). United Kingdom Official
Publication.
\end{Source}
\begin{References}\relax
Boot and Maindonald. New estimates of age- and sex-specific earnings,
and the male-female earnings gap in the British cotton industry,
1833-1906.  Unpublished manuscript.
\end{References}
\begin{Examples}
\begin{ExampleCode}
data(cottonworkers)
str(cottonworkers)
plot(survey1889 ~ census1886, data=cottonworkers)
plot(I(avwage*survey1889) ~ I(avwage*census1886), data=cottonworkers)
\end{ExampleCode}
\end{Examples}

