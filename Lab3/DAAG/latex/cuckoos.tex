\HeaderA{cuckoos}{Cuckoo Eggs Data}{cuckoos}
\keyword{datasets}{cuckoos}
\begin{Description}\relax
Length and breadth measurements of 120 eggs lain in the nests of six 
different species of host bird.
\end{Description}
\begin{Usage}
\begin{verbatim}cuckoos\end{verbatim}
\end{Usage}
\begin{Format}\relax
This data frame contains the following columns:
\describe{
\item[length] the egg lengths in millimeters
\item[breadth] the egg breadths in millimeters
\item[species] a factor with levels
\code{hedge.sparrow}, 
\code{meadow.pipit}, 
\code{pied.wagtail}, 
\code{robin}, 
\code{tree.pipit}, 
\code{wren} 
 
\item[id] a numeric vector
}
\end{Format}
\begin{Source}\relax
Latter, O.H. (1902). The eggs of Cuculus canorus. An
Inquiry into the dimensions of the cuckoo's egg and the relation of the
variations to the size of the eggs of the foster-parent, with notes on
coloration, \&c. Biometrika i, 164.
\end{Source}
\begin{References}\relax
Tippett, L.H.C. 1931: "The Methods of Statistics". Williams \& Norgate, 
London.
\end{References}
\begin{Examples}
\begin{ExampleCode} 
print("Strip and Boxplots - Example 2.1.2")

attach(cuckoos)
oldpar <- par(las = 2) # labels at right angle to axis.
stripchart(length ~ species) 
boxplot(split(cuckoos$length, cuckoos$species),
         xlab="Length of egg", horizontal=TRUE)
detach(cuckoos)
par(oldpar)
pause()

print("Summaries - Example 2.2.2")
sapply(split(cuckoos$length, cuckoos$species), sd)
pause()

print("Example 4.1.4")
wren <- split(cuckoos$length, cuckoos$species)$wren
median(wren)
n <- length(wren)
sqrt(pi/2)*sd(wren)/sqrt(n)  # this s.e. computation assumes normality
\end{ExampleCode}
\end{Examples}

