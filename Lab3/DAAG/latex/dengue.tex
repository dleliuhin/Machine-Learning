\HeaderA{dengue}{Dengue prevalence, by administrative region}{dengue}
\keyword{datasets}{dengue}
\begin{Description}\relax
Data record, for each of 2000 administrative regions,
whether or not dengue was recorded at any time between 1961 and 1990.
\end{Description}
\begin{Usage}
\begin{verbatim}data(dengue)\end{verbatim}
\end{Usage}
\begin{Format}\relax
A data frame with 2000 observations on the following 13 variables.
\describe{
\item[humid] Average vapour density: 1961-1990
\item[humid90] 90th percentile of \code{humid}
\item[temp] Average temperature: 1961-1990
\item[temp90] 90th percentile of \code{temp}
\item[h10pix] maximum of \texttt{humid}, within a 10 pixel radius
\item[h10pix90] maximum of \texttt{humid90}, within a 10 pixel radius
\item[trees] Percent tree cover, from satellite data
\item[trees90] 90th percentile of \texttt{trees}
\item[NoYes] Was dengue observed? (1=yes)
\item[Xmin] minimum longitude
\item[Xmax] maximum longitude
\item[Ymin] minimum latitude
\item[Ymax] maximum latitude
}
\end{Format}
\begin{Details}\relax
This is derived from a data set in which the climate and tree cover
information were given for each half degree of latitude by half
degreee of longitude pixel.
The variable \code{NoYes} was given by administrative region.
The climate data and tree cover data given here are 50th or 90th
percentiles, where percetiles were calculates across pixels for an
administrative region.
\end{Details}
\begin{Source}\relax
Simon Hales, Environmental Research New Zealand Ltd.
\end{Source}
\begin{References}\relax
Hales, S., de Wet, N., Maindonald, J. and Woodward, A.
2002.  Potential effect of population and climate change global
distribution of dengue fever: an empirical model.  The Lancet 2002;
360: 830-34.
\end{References}
\begin{Examples}
\begin{ExampleCode}
str(dengue)
glm(NoYes ~ humid, data=dengue, family=binomial)
glm(NoYes ~ humid90, data=dengue, family=binomial)
\end{ExampleCode}
\end{Examples}

