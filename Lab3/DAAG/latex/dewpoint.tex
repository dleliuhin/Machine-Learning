\HeaderA{dewpoint}{Dewpoint Data}{dewpoint}
\keyword{datasets}{dewpoint}
\begin{Description}\relax
The \code{dewpoint} data frame has 72 rows and 3 columns.
Monthly data were obtained for a number of sites (in Australia)
and a number of months.
\end{Description}
\begin{Usage}
\begin{verbatim}dewpoint\end{verbatim}
\end{Usage}
\begin{Format}\relax
This data frame contains the following columns:
\describe{
\item[maxtemp] monthly minimum temperatures
\item[mintemp] monthly maximum temperatures
\item[dewpt] monthly average dewpoint for each combination of
minimum and maximum temperature readings (formerly dewpoint)
}
\end{Format}
\begin{Source}\relax
Dr Edward Linacre,
visiting fellow in the Australian National University Department
of Geography.
\end{Source}
\begin{Examples}
\begin{ExampleCode}
print("Additive Model - Example 7.5")
require(splines)
attach(dewpoint)   
ds.lm <- lm(dewpt ~ bs(maxtemp,5) + bs(mintemp,5), data=dewpoint)
ds.fit <-predict(ds.lm, type="terms", se=TRUE)
oldpar <- par(mfrow=c(1,2))
plot(maxtemp, ds.fit$fit[,1], xlab="Maximum temperature",
     ylab="Change from dewpoint mean",type="n")
lines(maxtemp,ds.fit$fit[,1])
lines(maxtemp,ds.fit$fit[,1]-2*ds.fit$se[,1],lty=2)
lines(maxtemp,ds.fit$fit[,1]+2*ds.fit$se[,1],lty=2)
plot(mintemp,ds.fit$fit[,2],xlab="Minimum temperature",
     ylab="Change from dewpoint mean",type="n")
ord<-order(mintemp)
lines(mintemp[ord],ds.fit$fit[ord,2])
lines(mintemp[ord],ds.fit$fit[ord,2]-2*ds.fit$se[ord,2],lty=2)
lines(mintemp[ord],ds.fit$fit[ord,2]+2*ds.fit$se[ord,2],lty=2)
detach(dewpoint)
par(oldpar)

\end{ExampleCode}
\end{Examples}

