\HeaderA{elastic2}{Elastic Band Data Replicated Again}{elastic2}
\keyword{datasets}{elastic2}
\begin{Description}\relax
The \code{elastic2} data frame has 9 rows and 2 columns
giving, for each amount by which an elastic band is stretched
over the end of a ruler, the distance that the band traveled when
released.
\end{Description}
\begin{Usage}
\begin{verbatim}elastic2\end{verbatim}
\end{Usage}
\begin{Format}\relax
This data frame contains the following columns:
\describe{
\item[stretch] the amount by which the
elastic band was stretched
\item[distance] the distance traveled
}
\end{Format}
\begin{Source}\relax
J. H. Maindonald
\end{Source}
\begin{Examples}
\begin{ExampleCode}
plot(elastic2)
pause()

print("Chapter 5 Exercise")

yrange <- range(c(elastic1$distance, elastic2$distance))
xrange <- range(c(elastic1$stretch, elastic2$stretch))
plot(distance ~ stretch, data = elastic1, pch = 16, ylim = yrange, xlim = 
xrange)
points(distance ~ stretch, data = elastic2, pch = 15, col = 2)
legend(xrange[1], yrange[2], legend = c("Data set 1", "Data set 2"), pch = 
c(16, 15), col = c(1, 2))

elastic1.lm <- lm(distance ~ stretch, data = elastic1)
elastic2.lm <- lm(distance ~ stretch, data = elastic2)
abline(elastic1.lm)
abline(elastic2.lm, col = 2)
summary(elastic1.lm)
summary(elastic2.lm)
pause()

predict(elastic1.lm, se.fit=TRUE)
predict(elastic2.lm, se.fit=TRUE)
\end{ExampleCode}
\end{Examples}

