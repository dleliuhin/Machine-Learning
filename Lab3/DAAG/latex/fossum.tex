\HeaderA{fossum}{Female Possum Measurements}{fossum}
\keyword{datasets}{fossum}
\begin{Description}\relax
The \code{fossum} data frame consists of nine morphometric
measurements on each of 43 female mountain brushtail possums, trapped
at seven sites from Southern Victoria to central Queensland.
This is a subset of the \code{possum} data frame.
\end{Description}
\begin{Usage}
\begin{verbatim}fossum\end{verbatim}
\end{Usage}
\begin{Format}\relax
This data frame contains the following columns:
\describe{
\item[case] observation number
\item[site] one of seven locations where possums were
trapped
\item[Pop] a factor which classifies the sites as
\code{Vic} Victoria,
\code{other} New South Wales or Queensland
\item[sex] a factor with levels
\code{f} female,
\code{m} male 
\item[age] age
\item[hdlngth] head length
\item[skullw] skull width
\item[totlngth] total length
\item[taill] tail length
\item[footlgth] foot length
\item[earconch] ear conch length
\item[eye] distance from medial canthus to lateral canthus of
right eye
\item[chest] chest girth (in cm)
\item[belly] belly girth (in cm)
}
\end{Format}
\begin{Source}\relax
Lindenmayer, D. B., Viggers, K. L., Cunningham, R. B., and
Donnelly, C. F. 1995. Morphological variation among columns of the
mountain brushtail possum, Trichosurus caninus Ogilby
(Phalangeridae: Marsupiala). Australian Journal of Zoology 43:
449-458.
\end{Source}
\begin{Examples}
\begin{ExampleCode}
boxplot(fossum$totlngth)
\end{ExampleCode}
\end{Examples}

