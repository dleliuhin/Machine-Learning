\HeaderA{fruitohms}{Electrical Resistance of Kiwi Fruit}{fruitohms}
\keyword{datasets}{fruitohms}
\begin{Description}\relax
Data are from a study that examined how the electrical
resistance of a slab of kiwifruit changed with the apparent
juice content.
\end{Description}
\begin{Usage}
\begin{verbatim}fruitohms\end{verbatim}
\end{Usage}
\begin{Format}\relax
This data frame contains the following columns:
\describe{
\item[juice] apparent juice content (percent) 
\item[ohms] electrical resistance (in ohms)
}
\end{Format}
\begin{Source}\relax
Harker, F. R. and Maindonald J.H. 1994. Ripening of nectarine
fruit. Plant Physiology 106: 165 - 171.
\end{Source}
\begin{Examples}
\begin{ExampleCode}
plot(ohms ~ juice, xlab="Apparent juice content (%)",ylab="Resistance (ohms)", data=fruitohms)
lines(lowess(fruitohms$juice, fruitohms$ohms), lwd=2)
pause()

require(splines)
attach(fruitohms)
plot(ohms ~ juice, cex=0.8, xlab="Apparent juice content (%)",
     ylab="Resistance (ohms)", type="n")
fruit.lmb4 <- lm(ohms ~ bs(juice,4))
ord <- order(juice)
lines(juice[ord], fitted(fruit.lmb4)[ord], lwd=2)
ci <- predict(fruit.lmb4, interval="confidence")
lines(juice[ord], ci[ord,"lwr"])
lines(juice[ord], ci[ord,"upr"])
\end{ExampleCode}
\end{Examples}

