\HeaderA{headInjury}{Minor Head Injury (Simulated) Data}{headInjury}
\keyword{datasets}{headInjury}
\begin{Description}\relax
The \code{headInjury} data frame has 3121 rows and 11 columns.
The data were simulated according to a simple logistic
regression model to match roughly the clinical characteristics
of a sample of individuals who suffered minor head injuries.
\end{Description}
\begin{Usage}
\begin{verbatim}headInjury\end{verbatim}
\end{Usage}
\begin{Format}\relax
This data frame contains the following columns:
\describe{
\item[age.65] age factor (0 = under 65, 1 = over 65).
\item[amnesia.before] amnesia before impact (less than 30 minutes = 0, 
more 
than 30 minutes =1).
\item[basal.skull.fracture] (0 = no fracture, 1 = fracture).
\item[GCS.decrease] Glasgow Coma Scale decrease (0 = no 
deterioration, 1 = deterioration).
\item[GCS.13] initial Glasgow Coma Scale (0 = not `13', 1 = `13'). 
\item[GCS.15.2hours] Glasgow Coma Scale after 2 hours
(0 = not `15', 1 = '15').
\item[high.risk] assessed by clinician as high risk for neurological
intervention (0 = not high risk, 1 = high risk).
\item[loss.of.consciousness] (0 = conscious, 1 = loss of 
consciousness).
\item[open.skull.fracture] (0 = no fracture, 1 = fracture) 
\item[vomiting] (0 = no vomiting, 1 = vomiting)
\item[clinically.important.brain.injury] any acute brain finding
revealed on CT (0 = not present, 1 = present).
}
\end{Format}
\begin{References}\relax
Stiell, I.G., Wells, G.A., Vandemheen, K., Clement, C., Lesiuk, H.,
Laupacis, A., McKnight, R.D., Verbee, R., Brison, R., Cass, D., 
Eisenhauer, M., Greenberg, G.H., and Worthington, J. (2001) 
The Canadian CT Head Rule for Patients with Minor Head Injury,
The Lancet. 357: 1391-1396.
\end{References}

