\HeaderA{hills}{Scottish Hill Races Data}{hills}
\keyword{datasets}{hills}
\begin{Description}\relax
The record times in 1984 for 35 Scottish hill races.
\end{Description}
\begin{Usage}
\begin{verbatim}hills\end{verbatim}
\end{Usage}
\begin{Format}\relax
This data frame contains the following columns:
\describe{
\item[dist] distance, in miles (on the map)
\item[climb] total height gained during the route, in feet
\item[time] record time in hours
}
\end{Format}
\begin{Source}\relax
A.C. Atkinson (1986) Comment: Aspects of diagnostic regression
analysis. Statistical Science  1, 397-402.

Also, in MASS library, with time in minutes.
\end{Source}
\begin{References}\relax
A.C. Atkinson (1988) Transformations unmasked. Technometrics 30,
311-318. [ "corrects" the time for Knock Hill from 78.65 to 18.65. It   
is unclear if this based on the original records.]
\end{References}
\begin{Examples}
\begin{ExampleCode}
print("Transformation - Example 6.4.3")
pairs(hills, labels=c("dist\n\n(miles)", "climb\n\n(feet)", 
"time\n\n(hours)"))
pause()

pairs(log(hills), labels=c("dist\n\n(log(miles))", "climb\n\n(log(feet))",
  "time\n\n(log(hours))"))
pause()

hills0.loglm <- lm(log(time) ~ log(dist) + log(climb), data = hills)  
oldpar <- par(mfrow=c(2,2))
plot(hills0.loglm)
pause()

hills.loglm <- lm(log(time) ~ log(dist) + log(climb), data = hills[-18,])
summary(hills.loglm) 
plot(hills.loglm)
pause()

hills2.loglm <- lm(log(time) ~ log(dist)+log(climb)+log(dist):log(climb), 
data=hills[-18,])
anova(hills.loglm, hills2.loglm)
pause()

step(hills2.loglm)
pause()

summary(hills.loglm, corr=TRUE)$coef
pause()

summary(hills2.loglm, corr=TRUE)$coef
par(oldpar)
pause()

print("Nonlinear - Example 6.9.4")
hills.nls0 <- nls(time ~ (dist^alpha)*(climb^beta), start =
   c(alpha = .909, beta = .260), data = hills[-18,])
summary(hills.nls0)
plot(residuals(hills.nls0) ~ predict(hills.nls0)) # residual plot
pause()

hills$climb.mi <- hills$climb/5280
hills.nls <- nls(time ~ alpha + beta*dist + gamma*(climb.mi^delta),
  start=c(alpha = 1, beta = 1, gamma = 1, delta = 1), data=hills[-18,])
summary(hills.nls)
plot(residuals(hills.nls) ~ predict(hills.nls)) # residual plot

\end{ExampleCode}
\end{Examples}

