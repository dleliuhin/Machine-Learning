\HeaderA{ironslag}{Iron Content Measurements}{ironslag}
\keyword{datasets}{ironslag}
\begin{Description}\relax
The \code{ironslag} data frame has 53 rows and 2 columns.
Two methods for measuring the iron content in samples of slag 
were compared, a chemical and a magnetic method.  The chemical
method requires greater effort than the magnetic method.
\end{Description}
\begin{Usage}
\begin{verbatim}ironslag\end{verbatim}
\end{Usage}
\begin{Format}\relax
This data frame contains the following columns:
\describe{
\item[chemical] a numeric vector containing the measurements
coming from the chemical method
\item[magnetic] a numeric vector containing the measurments
coming from the magnetic method
}
\end{Format}
\begin{Source}\relax
Hand, D.J., Daly, F., McConway, K., Lunn, D., and Ostrowski, E. eds (1993)
A Handbook of Small Data Sets. London: Chapman \& Hall.
\end{Source}
\begin{Examples}
\begin{ExampleCode}
iron.lm <- lm(chemical ~ magnetic, data = ironslag)
oldpar <- par(mfrow = c(2,2))
plot(iron.lm)
par(oldpar)
\end{ExampleCode}
\end{Examples}

