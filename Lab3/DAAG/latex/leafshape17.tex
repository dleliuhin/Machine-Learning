\HeaderA{leafshape17}{Subset of Leaf Shape Data Set}{leafshape17}
\keyword{datasets}{leafshape17}
\begin{Description}\relax
The \code{leafshape17} data frame has 61 rows and 8 columns.
These are leaf length, width and petiole measurements taken
at several sites in Australia.  This is a subset of the 
\code{leafshape} data frame.
\end{Description}
\begin{Usage}
\begin{verbatim}leafshape17\end{verbatim}
\end{Usage}
\begin{Format}\relax
This data frame contains the following columns:
\describe{
\item[bladelen] leaf length (in mm)
\item[petiole] a numeric vector
\item[bladewid] leaf width (in mm)
\item[latitude] latitude
\item[logwid] natural logarithm of width
\item[logpet] logarithm of petiole measurement
\item[loglen] logarithm of length
\item[arch] leaf architecture (0 = orthotropic, 1 = plagiotropic)
}
\end{Format}
\begin{Source}\relax
King, D.A. and Maindonald, J.H. 1999. Tree architecture in relation to
leaf dimensions and tree stature in temperate and tropical rain
forests. Journal of Ecology 87: 1012-1024.
\end{Source}
\begin{Examples}
\begin{ExampleCode}
print("Discriminant Analysis - Example 11.2")

require(MASS)
leaf17.lda <- lda(arch ~ logwid+loglen, data=leafshape17)
leaf17.hat <- predict(leaf17.lda)
leaf17.lda
 table(leafshape17$arch, leaf17.hat$class)
pause()

tab <- table(leafshape17$arch, leaf17.hat$class)
 sum(tab[row(tab)==col(tab)])/sum(tab)
leaf17cv.lda <- lda(arch ~ logwid+loglen, data=leafshape17, CV=TRUE)
tab <- table(leafshape17$arch, leaf17cv.lda$class)
pause()

leaf17.glm <- glm(arch ~ logwid + loglen, family=binomial, data=leafshape17)
 options(digits=3)
summary(leaf17.glm)$coef
pause()

leaf17.one <- cv.binary(leaf17.glm)
table(leafshape17$arch, round(leaf17.one$internal))     # Resubstitution
pause()

table(leafshape17$arch, round(leaf17.one$cv))           # Cross-validation
\end{ExampleCode}
\end{Examples}

