\HeaderA{leaftemp}{Leaf and Air Temperature Data}{leaftemp}
\keyword{datasets}{leaftemp}
\begin{Description}\relax
These data consist of measurements of vapour pressure and of the 
difference between leaf and air temperature.
\end{Description}
\begin{Usage}
\begin{verbatim}leaftemp\end{verbatim}
\end{Usage}
\begin{Format}\relax
This data frame contains the following columns:
\describe{
\item[CO2level] Carbon Dioxide level
\code{low}, \code{medium}, \code{high} 
\item[vapPress] Vapour pressure
\item[tempDiff] Difference between leaf and air temperature
\item[BtempDiff] a numeric vector
}
\end{Format}
\begin{Source}\relax
Katharina Siebke and Susan von Cammerer, Australian National University.
\end{Source}
\begin{Examples}
\begin{ExampleCode}
print("Fitting Multiple Lines - Example 7.3")

leaf.lm1 <- lm(tempDiff ~ 1 , data = leaftemp)
leaf.lm2 <- lm(tempDiff ~ vapPress, data = leaftemp)
leaf.lm3 <- lm(tempDiff ~ CO2level + vapPress, data = leaftemp)
leaf.lm4 <- lm(tempDiff ~ CO2level + vapPress + vapPress:CO2level,
  data = leaftemp)

anova(leaf.lm1, leaf.lm2, leaf.lm3, leaf.lm4)

summary(leaf.lm2)
plot(leaf.lm2)

\end{ExampleCode}
\end{Examples}

