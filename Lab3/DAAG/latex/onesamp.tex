\HeaderA{onesamp}{Paired Sample t-test}{onesamp}
\keyword{models}{onesamp}
\begin{Description}\relax
This function performs a t-test for the mean difference for paired data,
and produces a scatterplot of one column against the other column, showing 
whether there was any benefit to using the paired design.
\end{Description}
\begin{Usage}
\begin{verbatim}
onesamp(dset=corn, x="unsprayed", y="sprayed", xlab=NULL, ylab=NULL, dubious=NULL, conv=NULL, dig=2)
\end{verbatim}
\end{Usage}
\begin{Arguments}
\begin{ldescription}
\item[\code{dset}] a matrix or dataframe having two columns
\item[\code{x}] name of column to play the role of the `predictor' 
\item[\code{y}] name of column to play the role of the `response' 
\item[\code{xlab}] horizontal axis label 
\item[\code{ylab}] vertical axis label 
\item[\code{dubious}] 
\item[\code{conv}] 
\item[\code{dig}] 
\end{ldescription}
\end{Arguments}
\begin{Value}
A scatterplot of \code{y} against \code{x} together with estimates
of standard errors and standard errors of the difference 
(\code{y}-\code{x}).

Also produced is a confidence interval and p-value for the test.
\end{Value}
\begin{Author}\relax
J.H. Maindonald
\end{Author}
\begin{Examples}
\begin{ExampleCode}
onesamp(dset = pair65, x = "ambient", y = "heated", xlab =
        "Amount of stretch (ambient)", ylab =
        "Amount of stretch (heated)") 
\end{ExampleCode}
\end{Examples}

