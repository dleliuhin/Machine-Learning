\HeaderA{ozone}{Ozone Data}{ozone}
\keyword{datasets}{ozone}
\begin{Description}\relax
Monthly provisional mean total ozone (in Dobson units) at Halley Bay 
(approximately corrected to Bass-Paur).
\end{Description}
\begin{Usage}
\begin{verbatim}ozone\end{verbatim}
\end{Usage}
\begin{Format}\relax
This data frame contains the following columns:
\describe{
\item[Year] the year
\item[Aug] August mean total ozone
\item[Sep] September mean total ozone
\item[Oct] October mean total ozone
\item[Nov] November mean total ozone
\item[Dec] December mean total ozone
\item[Jan] January mean total ozone
\item[Feb] February mean total ozone
\item[Mar] March mean total ozone
\item[Apr] April mean total ozone
\item[Annual] Yearly mean total ozone
}
\end{Format}
\begin{Source}\relax
Shanklin, J. (2001) Ozone at Halley, Rothera and Vernadsky/Faraday.  

http://www.antarctica.ac.uk/met/jds/ozone/data/zoz5699.dat
\end{Source}
\begin{References}\relax
Christie, M. (2000) The Ozone Layer: a Philosophy of Science Perspective.
Cambridge University Press.
\end{References}
\begin{Examples}
\begin{ExampleCode}
AnnualOzone <- ts(ozone$Annual, start=1956)
plot(AnnualOzone)
\end{ExampleCode}
\end{Examples}

