\HeaderA{possum}{Possum Measurements}{possum}
\keyword{datasets}{possum}
\begin{Description}\relax
The \code{possum} data frame consists of nine morphometric
measurements on each of 104 mountain brushtail possums, trapped
at seven sites from Southern Victoria to central Queensland.
\end{Description}
\begin{Usage}
\begin{verbatim}possum\end{verbatim}
\end{Usage}
\begin{Format}\relax
This data frame contains the following columns:
\describe{
\item[case] observation number
\item[site] one of seven locations where possums were
trapped
\item[Pop] a factor which classifies the sites as
\code{Vic} Victoria,
\code{other} New South Wales or Queensland
\item[sex] a factor with levels
\code{f} female,
\code{m} male 
\item[age] age
\item[hdlngth] head length
\item[skullw] skull width
\item[totlngth] total length
\item[taill] tail length
\item[footlgth] foot length
\item[earconch] ear conch length
\item[eye] distance from medial canthus to lateral canthus of
right eye
\item[chest] chest girth (in cm)
\item[belly] belly girth (in cm)
}
\end{Format}
\begin{Source}\relax
Lindenmayer, D. B., Viggers, K. L., Cunningham, R. B., and
Donnelly, C. F. 1995. Morphological variation among columns of the
mountain brushtail possum, Trichosurus caninus Ogilby
(Phalangeridae: Marsupiala). Australian Journal of Zoology 43:
449-458.
\end{Source}
\begin{Examples}
\begin{ExampleCode}
boxplot(earconch~sex, data=possum)
pause()

sex <- as.integer(possum$sex)
oldpar <- par(oma=c(2,4,5,4))
pairs(possum[, c(9:11)], pch=c(0,2:7), col=c("red","blue"),
  labels=c("tail\nlength","foot\nlength","ear conch\nlength"))
chh <- par()$cxy[2]; xleg <- 0.05; yleg <- 1.04
oldpar <- par(xpd=TRUE)  
legend(xleg, yleg, c("Cambarville", "Bellbird", "Whian Whian  ",
  "Byrangery", "Conondale  ","Allyn River", "Bulburin"), pch=c(0,2:7),
  x.intersp=1, y.intersp=0.75, cex=0.8, xjust=0, bty="n", ncol=4)
text(x=0.2, y=yleg - 2.25*chh, "female", col="red", cex=0.8, bty="n")
text(x=0.75, y=yleg - 2.25*chh, "male", col="blue", cex=0.8, bty="n")
par(oldpar)
pause()

sapply(possum[,6:14], function(x)max(x,na.rm=TRUE)/min(x,na.rm=TRUE))
pause()

here <- na.omit(possum$footlgth)
possum.prc <- princomp(possum[here, 6:14])
pause()

plot(possum.prc$scores[,1] ~ possum.prc$scores[,2],
  col=c("red","blue")[as.numeric(possum$sex[here])],
  pch=c(0,2:7)[possum$site[here]], xlab = "PC1", ylab = "PC2")
  # NB: We have abbreviated the axis titles
chh <- par()$cxy[2]; xleg <- -15; yleg <- 20.5
oldpar <- par(xpd=TRUE)
legend(xleg, yleg, c("Cambarville", "Bellbird", "Whian Whian  ",
  "Byrangery", "Conondale  ","Allyn River", "Bulburin"), pch=c(0,2:7),
  x.intersp=1, y.intersp=0.75, cex=0.8, xjust=0, bty="n", ncol=4)
text(x=-9, y=yleg - 2.25*chh, "female", col="red", cex=0.8, bty="n")
summary(possum.prc, loadings=TRUE, digits=2)
par(oldpar)
pause()

require(MASS)
here <- !is.na(possum$footlgth)
possum.lda <- lda(site ~ hdlngth+skullw+totlngth+ taill+footlgth+
  earconch+eye+chest+belly, data=possum, subset=here)
options(digits=4)
possum.lda$svd   # Examine the singular values   
plot(possum.lda, dimen=3)
  # Scatterplot matrix - scores on 1st 3 canonical variates (Figure 11.4)
possum.lda
\end{ExampleCode}
\end{Examples}

