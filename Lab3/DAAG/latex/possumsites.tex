\HeaderA{possumsites}{Possum Sites}{possumsites}
\keyword{datasets}{possumsites}
\begin{Description}\relax
The \code{possumsites} data frame consists of latitudes, longitudes,
and altitudes for the seven sites from Southern Victoria to central Queensland
where the \code{possum} observations were made.
\end{Description}
\begin{Usage}
\begin{verbatim}possumsites\end{verbatim}
\end{Usage}
\begin{Format}\relax
This data frame contains the following columns:
\describe{
\item[latitude] a numeric vector
\item[longitude] a numeric vector
\item[altitude] in meters
}
\end{Format}
\begin{Source}\relax
Lindenmayer, D. B., Viggers, K. L., Cunningham, R. B., and
Donnelly, C. F. 1995. Morphological variation among columns of the
mountain brushtail possum, Trichosurus caninus Ogilby
(Phalangeridae: Marsupiala). Australian Journal of Zoology 43:
449-458.
\end{Source}
\begin{Examples}
\begin{ExampleCode}
require(oz)
oz(sections=c(3:5, 11:16))
attach(possumsites)
points(latitude, longitude, pch=16, col=2)
chw <- par()$cxy[1]
chh <- par()$cxy[2]
posval <- c(2,4,2,2,4,2,2)
text(latitude+(3-posval)*chw/4, longitude, row.names(possumsites), pos=posval)
\end{ExampleCode}
\end{Examples}

