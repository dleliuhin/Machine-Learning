\HeaderA{roller}{Lawn Roller Data}{roller}
\keyword{datasets}{roller}
\begin{Description}\relax
The \code{roller} data frame has 10 rows and 2 columns.
Different weights of roller were rolled over different parts
of a lawn, and the depression was recorded.
\end{Description}
\begin{Usage}
\begin{verbatim}roller\end{verbatim}
\end{Usage}
\begin{Format}\relax
This data frame contains the following columns:
\describe{
\item[weight] a numeric vector consisting of the roller weights
\item[depression] the depth of the depression made in the grass
under the roller
}
\end{Format}
\begin{Source}\relax
Stewart, K.M., Van Toor, R.F., Crosbie,
S.F. 1988. Control of grass grub (Coleoptera: Scarabaeidae) with
rollers of different design. N.Z. Journal of Experimental Agriculture
16: 141-150.
\end{Source}
\begin{Examples}
\begin{ExampleCode}
plot(roller)
roller.lm <- lm(depression ~ weight, data = roller)
plot(roller.lm, which = 4)
\end{ExampleCode}
\end{Examples}

