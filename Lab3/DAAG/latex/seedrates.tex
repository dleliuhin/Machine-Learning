\HeaderA{seedrates}{Barley Seeding Rate Data}{seedrates}
\keyword{datasets}{seedrates}
\begin{Description}\relax
The \code{seedrates} data frame has 5 rows and 2 columns on 
the effect of seeding rate of barley on yield.
\end{Description}
\begin{Usage}
\begin{verbatim}seedrates\end{verbatim}
\end{Usage}
\begin{Format}\relax
This data frame contains the following columns:
\describe{
\item[rate] the seeding rate
\item[grain] the number of grain per head of barley
}
\end{Format}
\begin{Source}\relax
McLeod, C.C. 1982.
Effect of rates of seeding on barley grown for grain. New Zealand 
Journal of Agriculture 10: 133-136.
\end{Source}
\begin{References}\relax
Maindonald J H 1992. Statistical design, analysis and presentation
issues. New Zealand Journal of Agricultural Research 35: 121-141.
\end{References}
\begin{Examples}
\begin{ExampleCode}
plot(grain~rate,data=seedrates,xlim=c(50,180),ylim=c(15.5,22),axes=FALSE)
new.df<-data.frame(rate=(2:8)*25)
seedrates.lm1<-lm(grain~rate,data=seedrates)
seedrates.lm2<-lm(grain~rate+I(rate^2),data=seedrates)
hat1<-predict(seedrates.lm1,newdata=new.df,interval="confidence")
hat2<-predict(seedrates.lm2,newdata=new.df,interval="confidence")
axis(1,at=new.df$rate); axis(2); box()
z1<-spline(new.df$rate, hat1[,"fit"]); z2<-spline(new.df$rate,   
hat2[,"fit"])
rate<-new.df$rate; lines(z1$x,z1$y)
lines(spline(rate,hat1[,"lwr"]),lty=1,col=3)
lines(spline(rate,hat1[,"upr"]),lty=1,col=3)
lines(z2$x,z2$y,lty=4)
lines(spline(rate,hat2[,"lwr"]),lty=4,col=3)
lines(spline(rate,hat2[,"upr"]),lty=4,col=3)\end{ExampleCode}
\end{Examples}

