\HeaderA{spam7}{Spam E-mail Data}{spam7}
\keyword{datasets}{spam7}
\begin{Description}\relax
The data consist of 4601 email items, of which 1813 items were identified
as spam.
\end{Description}
\begin{Usage}
\begin{verbatim}spam7\end{verbatim}
\end{Usage}
\begin{Format}\relax
This data frame contains the following columns:
\describe{
\item[crl.tot] total length of words in capitals
\item[dollar] number of occurrences of the \$ symbol
\item[bang] number of occurrences of the ! symbol
\item[money] number of occurrences of the word `money'
\item[n000] number of occurrences of the string `000'
\item[make] number of occurrences of the word `make'
\item[yesno] outcome variable, a factor with levels
\code{n} not spam,
\code{y} spam
}
\end{Format}
\begin{Source}\relax
George Forman, Hewlett-Packard Laboratories

These data are available from the University
of California at Irvine Repository of Machine Learning Databases
and Domain Theories. The address is:  http://www.ics.uci.edu/~Here
\end{Source}
\begin{Examples}
\begin{ExampleCode}
require(rpart)
spam.rpart <- rpart(formula = yesno ~ crl.tot + dollar + bang +
   money + n000 + make, data=spam7)
plot(spam.rpart)
text(spam.rpart)

\end{ExampleCode}
\end{Examples}

