\HeaderA{twotPermutation}{Two Sample Permutation Test}{twotPermutation}
\keyword{models}{twotPermutation}
\begin{Description}\relax
This function computes the p-value for the two sample
t-test using a permutation test.  The permutation density
can  also be plotted.
\end{Description}
\begin{Usage}
\begin{verbatim}
twotPermutation(x1=two65$ambient, x2=two65$heated, nsim=2000, plotit=TRUE)
\end{verbatim}
\end{Usage}
\begin{Arguments}
\begin{ldescription}
\item[\code{x1}] Sample 1
\item[\code{x2}] Sample 2
\item[\code{nsim}] Number of simulations
\item[\code{plotit}] If TRUE, the permutation density will be plotted
\end{ldescription}
\end{Arguments}
\begin{Details}\relax
Suppose we have n1 values in one group and n2 in a
second, with n = n1 + n2. The permutation distribution
results from taking all possible samples of n2 values from
the total of n values.
\end{Details}
\begin{Value}
The p-value for the test of the hypothesis that the mean of
\code{x1} differs from \code{x2}
\end{Value}
\begin{Author}\relax
J.H. Maindonald
\end{Author}
\begin{References}\relax
Good, P. 2000. Permutation Tests. Springer, New York.
\end{References}
\begin{Examples}
\begin{ExampleCode}
twotPermutation()
\end{ExampleCode}
\end{Examples}

