\HeaderA{vlt}{Video Lottery Terminal Data}{vlt}
\keyword{datasets}{vlt}
\begin{Description}\relax
Data on objects appearing in three windows on a video
lottery terminal, together with the prize payout (usually 0).
Observations were taken on two successive days in late 1994
at a hotel lounge north of Winnipeg, Manitoba.  Each observation
cost 25 cents (Canadian).  The game played was `Double Diamond'.
\end{Description}
\begin{Usage}
\begin{verbatim}vlt\end{verbatim}
\end{Usage}
\begin{Format}\relax
This data frame contains the following columns:
\describe{
\item[window1] object appearing in the first window.
\item[window2] object appearing in the second window.
\item[window3] object appearing in the third window.
\item[prize] cash prize awarded (in Canadian dollars).
\item[night] 1, if observation was taken on day 1; 2,
if observation was taken on day 2.
}
\end{Format}
\begin{Details}\relax
At each play, each of three windows shows one of 7 possible objects.
Apparently, the three windows are independent of each other, and
the objects should appear with equal probability across the three windows.  
The objects are coded as follows: blank (0), single bar (1), double
bar (2), triple bar (3), double diamond (5), cherries (6), and 
the numeral "7" (7).  

Prizes (in quarters) are awarded according to the following scheme:  
800 (5-5-5), 80 (7-7-7), 40 (3-3-3), 25 (2-2-2), 10 (1-1-1), 10 (6-6-6),
5 (2 "6"'s), 2 (1 "6") and 5 (any combination of "1", "2" and "3").
In addition, a "5" doubles any winning combination, e.g. (5-3-3) pays
80 and (5-3-5) pays 160.
\end{Details}
\begin{Source}\relax
Braun, W. J. (1995) An illustration of bootstrapping using
video lottery terminal data.  Journal of Statistics Education
http://www.amstat.org/publications/jse/v3n2/datasets.braun.html
\end{Source}
\begin{Examples}
\begin{ExampleCode}
     vlt.stk <- stack(vlt[,1:3])
     table(vlt.stk)
\end{ExampleCode}
\end{Examples}

